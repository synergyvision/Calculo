\documentclass[12pt,]{krantz}
\usepackage{lmodern}
\usepackage{amssymb,amsmath}
\usepackage{ifxetex,ifluatex}
\usepackage{fixltx2e} % provides \textsubscript
\ifnum 0\ifxetex 1\fi\ifluatex 1\fi=0 % if pdftex
  \usepackage[T1]{fontenc}
  \usepackage[utf8]{inputenc}
\else % if luatex or xelatex
  \ifxetex
    \usepackage{mathspec}
  \else
    \usepackage{fontspec}
  \fi
  \defaultfontfeatures{Ligatures=TeX,Scale=MatchLowercase}
\fi
% use upquote if available, for straight quotes in verbatim environments
\IfFileExists{upquote.sty}{\usepackage{upquote}}{}
% use microtype if available
\IfFileExists{microtype.sty}{%
\usepackage[]{microtype}
\UseMicrotypeSet[protrusion]{basicmath} % disable protrusion for tt fonts
}{}
\PassOptionsToPackage{hyphens}{url} % url is loaded by hyperref
\usepackage[unicode=true]{hyperref}
\PassOptionsToPackage{usenames,dvipsnames}{color} % color is loaded by hyperref
\hypersetup{
            pdftitle={Cálculo},
            pdfauthor={Synergy Vision},
            colorlinks=true,
            linkcolor=Maroon,
            citecolor=Blue,
            urlcolor=Blue,
            breaklinks=true}
\urlstyle{same}  % don't use monospace font for urls
\usepackage{natbib}
\bibliographystyle{apalike}
\usepackage{color}
\usepackage{fancyvrb}
\newcommand{\VerbBar}{|}
\newcommand{\VERB}{\Verb[commandchars=\\\{\}]}
\DefineVerbatimEnvironment{Highlighting}{Verbatim}{commandchars=\\\{\}}
% Add ',fontsize=\small' for more characters per line
\usepackage{framed}
\definecolor{shadecolor}{RGB}{248,248,248}
\newenvironment{Shaded}{\begin{snugshade}}{\end{snugshade}}
\newcommand{\KeywordTok}[1]{\textcolor[rgb]{0.13,0.29,0.53}{\textbf{#1}}}
\newcommand{\DataTypeTok}[1]{\textcolor[rgb]{0.13,0.29,0.53}{#1}}
\newcommand{\DecValTok}[1]{\textcolor[rgb]{0.00,0.00,0.81}{#1}}
\newcommand{\BaseNTok}[1]{\textcolor[rgb]{0.00,0.00,0.81}{#1}}
\newcommand{\FloatTok}[1]{\textcolor[rgb]{0.00,0.00,0.81}{#1}}
\newcommand{\ConstantTok}[1]{\textcolor[rgb]{0.00,0.00,0.00}{#1}}
\newcommand{\CharTok}[1]{\textcolor[rgb]{0.31,0.60,0.02}{#1}}
\newcommand{\SpecialCharTok}[1]{\textcolor[rgb]{0.00,0.00,0.00}{#1}}
\newcommand{\StringTok}[1]{\textcolor[rgb]{0.31,0.60,0.02}{#1}}
\newcommand{\VerbatimStringTok}[1]{\textcolor[rgb]{0.31,0.60,0.02}{#1}}
\newcommand{\SpecialStringTok}[1]{\textcolor[rgb]{0.31,0.60,0.02}{#1}}
\newcommand{\ImportTok}[1]{#1}
\newcommand{\CommentTok}[1]{\textcolor[rgb]{0.56,0.35,0.01}{\textit{#1}}}
\newcommand{\DocumentationTok}[1]{\textcolor[rgb]{0.56,0.35,0.01}{\textbf{\textit{#1}}}}
\newcommand{\AnnotationTok}[1]{\textcolor[rgb]{0.56,0.35,0.01}{\textbf{\textit{#1}}}}
\newcommand{\CommentVarTok}[1]{\textcolor[rgb]{0.56,0.35,0.01}{\textbf{\textit{#1}}}}
\newcommand{\OtherTok}[1]{\textcolor[rgb]{0.56,0.35,0.01}{#1}}
\newcommand{\FunctionTok}[1]{\textcolor[rgb]{0.00,0.00,0.00}{#1}}
\newcommand{\VariableTok}[1]{\textcolor[rgb]{0.00,0.00,0.00}{#1}}
\newcommand{\ControlFlowTok}[1]{\textcolor[rgb]{0.13,0.29,0.53}{\textbf{#1}}}
\newcommand{\OperatorTok}[1]{\textcolor[rgb]{0.81,0.36,0.00}{\textbf{#1}}}
\newcommand{\BuiltInTok}[1]{#1}
\newcommand{\ExtensionTok}[1]{#1}
\newcommand{\PreprocessorTok}[1]{\textcolor[rgb]{0.56,0.35,0.01}{\textit{#1}}}
\newcommand{\AttributeTok}[1]{\textcolor[rgb]{0.77,0.63,0.00}{#1}}
\newcommand{\RegionMarkerTok}[1]{#1}
\newcommand{\InformationTok}[1]{\textcolor[rgb]{0.56,0.35,0.01}{\textbf{\textit{#1}}}}
\newcommand{\WarningTok}[1]{\textcolor[rgb]{0.56,0.35,0.01}{\textbf{\textit{#1}}}}
\newcommand{\AlertTok}[1]{\textcolor[rgb]{0.94,0.16,0.16}{#1}}
\newcommand{\ErrorTok}[1]{\textcolor[rgb]{0.64,0.00,0.00}{\textbf{#1}}}
\newcommand{\NormalTok}[1]{#1}
\usepackage{longtable,booktabs}
% Fix footnotes in tables (requires footnote package)
\IfFileExists{footnote.sty}{\usepackage{footnote}\makesavenoteenv{long table}}{}
\usepackage{graphicx,grffile}
\makeatletter
\def\maxwidth{\ifdim\Gin@nat@width>\linewidth\linewidth\else\Gin@nat@width\fi}
\def\maxheight{\ifdim\Gin@nat@height>\textheight\textheight\else\Gin@nat@height\fi}
\makeatother
% Scale images if necessary, so that they will not overflow the page
% margins by default, and it is still possible to overwrite the defaults
% using explicit options in \includegraphics[width, height, ...]{}
\setkeys{Gin}{width=\maxwidth,height=\maxheight,keepaspectratio}
\IfFileExists{parskip.sty}{%
\usepackage{parskip}
}{% else
\setlength{\parindent}{0pt}
\setlength{\parskip}{6pt plus 2pt minus 1pt}
}
\setlength{\emergencystretch}{3em}  % prevent overfull lines
\providecommand{\tightlist}{%
  \setlength{\itemsep}{0pt}\setlength{\parskip}{0pt}}
\setcounter{secnumdepth}{5}
% Redefines (sub)paragraphs to behave more like sections
\ifx\paragraph\undefined\else
\let\oldparagraph\paragraph
\renewcommand{\paragraph}[1]{\oldparagraph{#1}\mbox{}}
\fi
\ifx\subparagraph\undefined\else
\let\oldsubparagraph\subparagraph
\renewcommand{\subparagraph}[1]{\oldsubparagraph{#1}\mbox{}}
\fi

% set default figure placement to htbp
\makeatletter
\def\fps@figure{htbp}
\makeatother

\usepackage[T1]{fontenc}
\usepackage[utf8]{inputenc} % recommended encoding
\usepackage[spanish]{babel}
\usepackage{booktabs}
\usepackage{longtable}
\usepackage[bf,singlelinecheck=off]{caption}

%\setmainfont[UprightFeatures={SmallCapsFont=AlegreyaSC-Regular}]{Alegreya}

\usepackage{framed,color}
\definecolor{shadecolor}{RGB}{248,248,248}

\renewcommand{\textfraction}{0.05}
\renewcommand{\topfraction}{0.8}
\renewcommand{\bottomfraction}{0.8}
\renewcommand{\floatpagefraction}{0.75}

\renewenvironment{quote}{\begin{VF}}{\end{VF}}
\let\oldhref\href
\renewcommand{\href}[2]{#2\footnote{\url{#1}}}

\ifxetex
  \usepackage{letltxmacro}
  \setlength{\XeTeXLinkMargin}{1pt}
  \LetLtxMacro\SavedIncludeGraphics\includegraphics
  \def\includegraphics#1#{% #1 catches optional stuff (star/opt. arg.)
    \IncludeGraphicsAux{#1}%
  }%
  \newcommand*{\IncludeGraphicsAux}[2]{%
    \XeTeXLinkBox{%
      \SavedIncludeGraphics#1{#2}%
    }%
  }%
\fi

\makeatletter
\newenvironment{kframe}{%
\medskip{}
\setlength{\fboxsep}{.8em}
 \def\at@end@of@kframe{}%
 \ifinner\ifhmode%
  \def\at@end@of@kframe{\end{minipage}}%
  \begin{minipage}{\columnwidth}%
 \fi\fi%
 \def\FrameCommand##1{\hskip\@totalleftmargin \hskip-\fboxsep
 \colorbox{shadecolor}{##1}\hskip-\fboxsep
     % There is no \\@totalrightmargin, so:
     \hskip-\linewidth \hskip-\@totalleftmargin \hskip\columnwidth}%
 \MakeFramed {\advance\hsize-\width
   \@totalleftmargin\z@ \linewidth\hsize
   \@setminipage}}%
 {\par\unskip\endMakeFramed%
 \at@end@of@kframe}
\makeatother

\renewenvironment{Shaded}{\begin{kframe}}{\end{kframe}}

\newenvironment{rmdblock}[1]
  {
  \begin{itemize}
  \renewcommand{\labelitemi}{
    \raisebox{-.7\height}[0pt][0pt]{
      {\setkeys{Gin}{width=3em,keepaspectratio}\includegraphics{images/#1}}
    }
  }
  \setlength{\fboxsep}{1em}
  \begin{kframe}
  \item
  }
  {
  \end{kframe}
  \end{itemize}
  }
\newenvironment{rmdnote}
  {\begin{rmdblock}{note}}
  {\end{rmdblock}}
\newenvironment{rmdcaution}
  {\begin{rmdblock}{caution}}
  {\end{rmdblock}}
\newenvironment{rmdimportant}
  {\begin{rmdblock}{important}}
  {\end{rmdblock}}
\newenvironment{rmdtip}
  {\begin{rmdblock}{tip}}
  {\end{rmdblock}}
\newenvironment{rmdwarning}
  {\begin{rmdblock}{warning}}
  {\end{rmdblock}}

\usepackage{makeidx}
\makeindex

\urlstyle{tt}

\usepackage{amsthm}
\makeatletter
\def\thm@space@setup{%
  \thm@preskip=8pt plus 2pt minus 4pt
  \thm@postskip=\thm@preskip
}
\makeatother

\frontmatter

\title{Cálculo}
\providecommand{\subtitle}[1]{}
\subtitle{Ciencia de los Datos Financieros}
\author{Synergy Vision}
\date{2018-05-07}

\usepackage{amsthm}
\newtheorem{theorem}{Teorema}[chapter]
\newtheorem{lemma}{Lema}[chapter]
\theoremstyle{definition}
\newtheorem{definition}{Definición}[chapter]
\newtheorem{corollary}{Corolario}[chapter]
\newtheorem{proposition}{Proposición}[chapter]
\theoremstyle{definition}
\newtheorem{example}{Ejemplo}[chapter]
\theoremstyle{definition}
\newtheorem{exercise}{Ejercicio}[chapter]
\theoremstyle{remark}
\newtheorem*{remark}{Nota}
\newtheorem*{solution}{Solución}
\let\BeginKnitrBlock\begin \let\EndKnitrBlock\end
\begin{document}
\maketitle

%\cleardoublepage\newpage\thispagestyle{empty}\null
%\cleardoublepage\newpage\thispagestyle{empty}\null
%\cleardoublepage\newpage
\thispagestyle{empty}

\setlength{\abovedisplayskip}{-5pt}
\setlength{\abovedisplayshortskip}{-5pt}

{
\hypersetup{linkcolor=black}
\setcounter{tocdepth}{2}
\tableofcontents
}
\listoftables
\listoffigures
\chapter*{Prefacio}\label{prefacio}


\includegraphics{images/by-nc-sa.png}\\
La versión en línea de este libro se comparte bajo la licencia
\href{http://creativecommons.org/licenses/by-nc-sa/4.0/}{Creative
Commons Attribution-NonCommercial-ShareAlike 4.0 International License}.

\section*{¿Por qué leer este libro?}\label{por-que-leer-este-libro}


Este libro es el resultado de enfocarnos en proveer la mayor cantidad de
material sobre Cálculo con un desarrollo teórico lo más explícito
posible, con el valor agregado de incorporar ejemplos de las finanzas y
la programación en \texttt{R}. Finalmente tenemos un libro interactivo
que ofrece una experiencia de aprendizaje distinta e innovadora.

Es mucha la literatura, pero son pocas las opciones donde se pueda
navegar el libro de forma amigable y además contar con ejemplos en
\texttt{R} y ejercicios interactivos, además del contenido multimedia.
Esperamos que ésta sea un contribución sobre nuevas prácticas para
publicar el contenido y darle vida, crear una experiencia distinta, una
experiencia interactiva y visual. El reto es darle vida al contenido
asistidos con las herramientas de Internet.

Finalmente este es un intento de ofrecer otra visión sobre la enseñanza
y la generación de material más accesible. Estamos en un mundo
multidisciplinado, es por ello que ahora hay que generar contenido que
conjugue en un mismo lugar las matemáticas, estadística, finanzas y la
computación.

Lo dejamos público ya que las herramientas que usamos para ensamblarlo
son abiertas y públicas.

\section*{Estructura del libro}\label{estructura-del-libro}


\section*{Información sobre los programas y
convenciones}\label{informacion-sobre-los-programas-y-convenciones}
\addcontentsline{toc}{section}{Información sobre los programas y
convenciones}

Este libro es posible gracias a una gran cantidad de desarrolladores que
contribuyen en la construcción de herramientas para generar documentos
enriquecidos e interactivos. En particular al autor de los paquetes
Yihui Xie xie2015.

\section*{Prácticas interactivas con
R}\label{practicas-interactivas-con-r}


Vamos a utilizar el paquete
\href{https://github.com/datacamp/tutorial}{Datacamp Tutorial} que
utiliza la librería en JavaScript
\href{https://github.com/datacamp/datacamp-light}{Datacamp Light} para
crear ejercicios y prácticas con \texttt{R}. De esta forma el libro es
completamente interactivo y con prácticas incluidas. De esta forma
estamos creando una experiencia única de aprendizaje en línea.

eyJsYW5ndWFnZSI6InIiLCJwcmVfZXhlcmNpc2VfY29kZSI6ImIgPC0gNSIsInNhbXBsZSI6IiMgQ3JlYSB1bmEgdmFyaWFibGUgYSwgaWd1YWwgYSA1XG5cbiMgTXVlc3RyYSBlbCB2YWxvciBkZSBhIiwic29sdXRpb24iOiIjIENyZWEgdW5hIHZhcmlhYmxlIGEsIGlndWFsIGEgNVxuYSA8LSA1XG5cbiMgTXVlc3RyYSBlbCB2YWxvciBkZSBhXG5hIiwic2N0IjoidGVzdF9vYmplY3QoXCJhXCIpXG50ZXN0X291dHB1dF9jb250YWlucyhcImFcIiwgaW5jb3JyZWN0X21zZyA9IFwiQXNlZ3VyYXRlIGRlIG1vc3RyYXIgZWwgdmFsb3IgZGUgYGFgLlwiKVxuc3VjY2Vzc19tc2coXCJFeGNlbGVudGUhXCIpIn0=

\section*{Agradecimientos}\label{agradecimientos}


A todo el equipo de Synergy Vision que no deja de soñar. Hay que hacer
lo que pocos hacen, insistir, insistir hasta alcanzar. Lo más importante
es concretar las ideas. La idea es sólo el inicio y solo vale cuando se
concreta.

\BeginKnitrBlock{flushright}
Synergy Vision, Caracas, Venezuela
\EndKnitrBlock{flushright}

\chapter*{Acerca del Autor}\label{acerca-del-autor}


Este material es un esfuerzo de equipo en Synergy Vision,
(\url{http://synergy.vision/nosotros/}).

El propósito de este material es ofrecer una experiencia de aprendizaje
distinta y enfocada en el estudiante. El propósito es que realmente
aprenda y practique con mucha intensidad. La idea es cambiar el modelo
de clases magistrales y ofrecer una experiencia más centrada en el
estudiante y menos centrado en el profesor. Para los temas más técnicos
y avanzados es necesario trabajar de la mano con el estudiante y
asistirlo en el proceso de aprendizaje con prácticas guiadas, material
en línea e interactivo, videos, evaluación contínua de brechas y
entendimiento, entre otros, para procurar el dominio de la materia.

Nuestro foco es la Ciencia de los Datos Financieros y para ello se
desarrollará material sobre: \textbf{Probabilidad y Estadística
Matemática en R}, \textbf{Programación Científica en R},
\textbf{Mercados}, \textbf{Inversiones y Trading}, \textbf{Datos y
Modelos Financieros en R}, \textbf{Renta Fija}, \textbf{Inmunización de
Carteras de Renta Fija}, \textbf{Teoría de Riesgo en R},
\textbf{Finanzas Cuantitativas}, \textbf{Ingeniería Financiera},
\textbf{Procesos Estocásticos en R}, \textbf{Series de Tiempo en R},
\textbf{Ciencia de los Datos}, \textbf{Ciencia de los Datos
Financieros}, \textbf{Simulación en R}, \textbf{Desarrollo de
Aplicaciones Interactivas en R}, \textbf{Minería de Datos},
\textbf{Aprendizaje Estadístico}, \textbf{Estadística Multivariante},
\textbf{Riesgo de Crédito}, \textbf{Riesgo de Liquidez}, \textbf{Riesgo
de Mercado}, \textbf{Riesgo Operacional}, \textbf{Riesgo de Cambio},
\textbf{Análisis Técnico}, \textbf{Inversión Visual}, \textbf{Finanzas},
\textbf{Finanzas Corporativas}, \textbf{Valoración}, \textbf{Teoría de
Portafolio}, entre otros.

Nuestra cuenta de Twitter es (\url{https://twitter.com/bysynergyvision})
y nuestros repositorios están en GitHub
(\url{https://github.com/synergyvision}).

\textbf{Somos Científicos de Datos Financieros}

\mainmatter

\chapter{Introducción}\label{introduccion}

\chapter{Funciones}\label{funciones}

\chapter{Límites}\label{limites}

\chapter{Derivadas}\label{derivadas}

\chapter{Integrales}\label{integrales}

\chapter{Aproximación de funciones}\label{aproximacion-de-funciones}

\chapter{Series}\label{series}

\cleardoublepage 

\appendix \addcontentsline{toc}{chapter}{\appendixname}


\chapter{Software Tools}\label{software-tools}

For those who are not familiar with software packages required for using
R Markdown, we give a brief introduction to the installation and
maintenance of these packages.

\section{R and R packages}\label{r-and-r-packages}

R can be downloaded and installed from any CRAN (the Comprehensive R
Archive Network) mirrors, e.g., \url{https://cran.rstudio.com}. Please
note that there will be a few new releases of R every year, and you may
want to upgrade R occasionally.

To install the \textbf{bookdown} package, you can type this in R:

\begin{Shaded}
\begin{Highlighting}[]
\KeywordTok{install.packages}\NormalTok{(}\StringTok{"bookdown"}\NormalTok{)}
\end{Highlighting}
\end{Shaded}

This installs all required R packages. You can also choose to install
all optional packages as well, if you do not care too much about whether
these packages will actually be used to compile your book (such as
\textbf{htmlwidgets}):

\begin{Shaded}
\begin{Highlighting}[]
\KeywordTok{install.packages}\NormalTok{(}\StringTok{"bookdown"}\NormalTok{, }\DataTypeTok{dependencies =} \OtherTok{TRUE}\NormalTok{)}
\end{Highlighting}
\end{Shaded}

If you want to test the development version of \textbf{bookdown} on
GitHub, you need to install \textbf{devtools} first:

\begin{Shaded}
\begin{Highlighting}[]
\ControlFlowTok{if}\NormalTok{ (}\OperatorTok{!}\KeywordTok{requireNamespace}\NormalTok{(}\StringTok{'devtools'}\NormalTok{)) }\KeywordTok{install.packages}\NormalTok{(}\StringTok{'devtools'}\NormalTok{)}
\NormalTok{devtools}\OperatorTok{::}\KeywordTok{install_github}\NormalTok{(}\StringTok{'rstudio/bookdown'}\NormalTok{)}
\end{Highlighting}
\end{Shaded}

R packages are also often constantly updated on CRAN or GitHub, so you
may want to update them once in a while:

\begin{Shaded}
\begin{Highlighting}[]
\KeywordTok{update.packages}\NormalTok{(}\DataTypeTok{ask =} \OtherTok{FALSE}\NormalTok{)}
\end{Highlighting}
\end{Shaded}

Although it is not required, the RStudio IDE can make a lot of things
much easier when you work on R-related projects. The RStudio IDE can be
downloaded from \url{https://www.rstudio.com}.

\section{Pandoc}\label{pandoc}

An R Markdown document (\texttt{*.Rmd}) is first compiled to Markdown
(\texttt{*.md}) through the \textbf{knitr} package, and then Markdown is
compiled to other output formats (such as LaTeX or HTML) through
Pandoc.\index{Pandoc} This process is automated by the
\textbf{rmarkdown} package. You do not need to install \textbf{knitr} or
\textbf{rmarkdown} separately, because they are the required packages of
\textbf{bookdown} and will be automatically installed when you install
\textbf{bookdown}. However, Pandoc is not an R package, so it will not
be automatically installed when you install \textbf{bookdown}. You can
follow the installation instructions on the Pandoc homepage
(\url{http://pandoc.org}) to install Pandoc, but if you use the RStudio
IDE, you do not really need to install Pandoc separately, because
RStudio includes a copy of Pandoc. The Pandoc version number can be
obtained via:

\begin{Shaded}
\begin{Highlighting}[]
\NormalTok{rmarkdown}\OperatorTok{::}\KeywordTok{pandoc_version}\NormalTok{()}
\NormalTok{## [1] '1.19.2.1'}
\end{Highlighting}
\end{Shaded}

If you find this version too low and there are Pandoc features only in a
later version, you can install the later version of Pandoc, and
\textbf{rmarkdown} will call the newer version instead of its built-in
version.

\section{LaTeX}\label{latex}

LaTeX\index{LaTeX} is required only if you want to convert your book to
PDF. The typical choice of the LaTeX distribution depends on your
operating system. Windows users may consider MiKTeX
(\url{http://miktex.org}), Mac OS X users can install MacTeX
(\url{http://www.tug.org/mactex/}), and Linux users can install TeXLive
(\url{http://www.tug.org/texlive}). See
\url{https://www.latex-project.org/get/} for more information about
LaTeX and its installation.

Most LaTeX distributions provide a minimal/basic package and a full
package. You can install the basic package if you have limited disk
space and know how to install LaTeX packages later. The full package is
often significantly larger in size, since it contains all LaTeX
packages, and you are unlikely to run into the problem of missing
packages in LaTeX.

LaTeX error messages may be obscure to beginners, but you may find
solutions by searching for the error message online (you have good
chances of ending up on
\href{http://tex.stackexchange.com}{StackExchange}). In fact, the LaTeX
code converted from R Markdown should be safe enough and you should not
frequently run into LaTeX problems unless you introduced raw LaTeX
content in your Rmd documents. The most common LaTeX problem should be
missing LaTeX packages, and the error may look like this:

\begin{Shaded}
\begin{Highlighting}[]
\NormalTok{! LaTeX Error: File `titling.sty' not found.}

\NormalTok{Type X to quit or <RETURN> to proceed,}
\NormalTok{or enter new name. (Default extension: sty)}

\NormalTok{Enter file name: }
\NormalTok{! Emergency stop.}
\NormalTok{<read *> }
         
\NormalTok{l.107 ^^M}

\NormalTok{pandoc: Error producing PDF}
\NormalTok{Error: pandoc document conversion failed with error 43}
\NormalTok{Execution halted}
\end{Highlighting}
\end{Shaded}

This means you used a package that contains \texttt{titling.sty}, but it
was not installed. LaTeX package names are often the same as the
\texttt{*.sty} filenames, so in this case, you can try to install the
\texttt{titling} package. Both MiKTeX and MacTeX provide a graphical
user interface to manage packages. You can find the MiKTeX package
manager from the start menu, and MacTeX's package manager from the
application ``TeX Live Utility''. Type the name of the package, or the
filename to search for the package and install it. TeXLive may be a
little trickier: if you use the pre-built TeXLive packages of your Linux
distribution, you need to search in the package repository and your
keywords may match other non-LaTeX packages. Personally, I find it
frustrating to use the pre-built collections of packages on Linux, and
much easier to install TeXLive from source, in which case you can manage
packages using the \texttt{tlmgr} command. For example, you can search
for \texttt{titling.sty} from the TeXLive package repository:

\begin{Shaded}
\begin{Highlighting}[]
\ExtensionTok{tlmgr}\NormalTok{ search --global --file titling.sty}
\CommentTok{# titling:}
\CommentTok{#    texmf-dist/tex/latex/titling/titling.sty}
\end{Highlighting}
\end{Shaded}

Once you have figured out the package name, you can install it by:

\begin{Shaded}
\begin{Highlighting}[]
\ExtensionTok{tlmgr}\NormalTok{ install titling  # may require sudo}
\end{Highlighting}
\end{Shaded}

LaTeX distributions and packages are also updated from time to time, and
you may consider updating them especially when you run into LaTeX
problems. You can find out the version of your LaTeX distribution by:

\begin{Shaded}
\begin{Highlighting}[]
\KeywordTok{system}\NormalTok{(}\StringTok{'pdflatex --version'}\NormalTok{)}
\NormalTok{## pdfTeX 3.14159265-2.6-1.40.18 (TeX Live 2017)}
\NormalTok{## kpathsea version 6.2.3}
\NormalTok{## Copyright 2017 Han The Thanh (pdfTeX) et al.}
\NormalTok{## There is NO warranty.  Redistribution of this software is}
\NormalTok{## covered by the terms of both the pdfTeX copyright and}
\NormalTok{## the Lesser GNU General Public License.}
\NormalTok{## For more information about these matters, see the file}
\NormalTok{## named COPYING and the pdfTeX source.}
\NormalTok{## Primary author of pdfTeX: Han The Thanh (pdfTeX) et al.}
\NormalTok{## Compiled with libpng 1.6.29; using libpng 1.6.29}
\NormalTok{## Compiled with zlib 1.2.11; using zlib 1.2.11}
\NormalTok{## Compiled with xpdf version 3.04}
\end{Highlighting}
\end{Shaded}

\bibliography{book.bib,packages.bib}

\backmatter
\printindex

\end{document}
